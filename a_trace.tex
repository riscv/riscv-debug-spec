\section{Trace Module}
\label{sec:a_trace}

{\bf This part of the spec needs work before it's ready to be implemented,
which is why it's in the appendix. It's left here to give a rough idea of some
of the issues to consider.}

Aside from viewing the current state of a core, knowing what happened in the
past can be incredibly helpful. Capturing an execution trace can give a user
that view.  Unfortunately processors run so fast that they generate trace data
at a very large rate. To help deal with this, the trace data format allows for
some simple compression.

The trace functionality described here aims to support 3 different use cases:
\begin{enumerate}
    \item Full reconstruction of all processor state, including register values
        etc. To achieve this goal the decoder will have to know what code is
        being executed, and know the exact behavior of every RISC-V
        instruction.
    \item Reconstruct just the instruction stream. Get enough data from the
        trace stream that it is possible to make a list of every instruction
        executed.  This is possible without knowing anything about the code or
        the core executing it.
    \item Watch memory accesses for a certain memory region.
\end{enumerate}

Trace data may be stored to a special on-core RAM, RAM on the system bus, or to
a dedicated off-chip interface. Only the system RAM destination is covered
here.

\subsection{Trace Data Format}

Trace data should be both compact and easy to generate. Ideally it's also easy
to decode, but since decoding doesn't have to happen in real time and will
usually have a powerful workstation to do the work, this is the least important
concern.

Trace data consists of a stream of 4-bit packets, which are stored in memory in
32-bit words by putting the first packet in bits 3:0 of the 32-bit word, the
second packet into bits 7:4, and so on. Trace packets and their encoding are
listed in Table~\ref{tab:tracepackets}.

\begin{table}[htp]
   \centering
   \caption{Trace Sequence Header Packets}
   \label{tab:tracepackets}
   \begin{tabulary}{\textwidth}{|l|l|L|}
      \hline
      0000 & Nop & Packet that indicates no data. The trace source must use
      these to ensure that there are 8 synchronization points in each buffer. \\
      \hline
      0001 & PC & Followed by a Value Sequence containing bits XLEN-1:1 of the
      PC if the compressed ISA is supported, or bits XLEN-1:2 of the PC if the
      compressed ISA is not supported.
      Missing bits must be filled in with the last PC value. \\
      \hline
      0010 & Branch Taken & \\
      \hline
      0011 & Branch Not Taken & \\
      \hline
      0100 & Trace Enabled & Followed by a single packet indicating the version
      of the trace data (currently 0). \\
      \hline
      0101 & Trace Disabled & Indicates that trace was purposefully disabled,
      or that some sequences were dropped because the trace buffer overflowed. \\
      \hline
      0110 & Privilege Level & Followed by a packet containing whether the
      cause of the change was an interrupt (1) or something else (0) in bit 3,
      PRV[1:0] in bits 2:1, and IE in bit 0. \\
      \hline
      0111 & Change Hart & Followed by a Value Sequence containing the hart ID
      of the hart whose trace data follows. Missing bits must be filled in with
      0. \\
      \hline
      1000 & Load Address & Followed by a Value Sequence containing the
      address.  Missing bits must be filled in with the last Load Address
      value. \\
      \hline
      1001 & Store Address & Followed by a Value Sequence containing the
      address. Missing bits must be filled in with the last Store Address
      value. \\
      \hline
      1010 & Load Data & Followed by a Value Sequence containing the data.
      Missing bits must be filled in by sign extending the value. \\
      \hline
      1011 & Store Data & Followed by a Value Sequence containing the data.
      Missing bits must be filled in by sign extending the value. \\
      \hline
      1100 & Timestamp & Followed by a Value Sequence containing the timestamp.
      Missing bits should be filled in with the last Timestamp value. \\
      \hline
      1101 & Reserved & Reserved for future standards. \\
      \hline
      1110 & Custom & Reserved for custom trace data. \\
      \hline
      1111 & Custom & Reserved for custom trace data. \\
      \hline
   \end{tabulary}
\end{table}

Several header packets are followed by a Value Sequence, which can encode
values between 4 and 64 bits. The sequence consists first of a 4-bit size
packet which contains a single number N.  It is followed by N+1 4-bit packets
which contain the value. The first packet contains bits 3:0 of the value. The
next packet contains bits 7:4, and so on.

\subsection{Trace Events}

Trace events are events that occur when a core is running that result in trace
packets being emitted. They are listed in Table~\ref{tab:traceevents}.

\begin{table}[htp]
   \centering
   \caption{Trace Data Events}
   \label{tab:traceevents}
   \begin{tabulary}{\textwidth}{|l|L|}
      \hline
      Opcode & Action \\
      \hline
      {\tt jal} & If \Femitbranch is disabled but \Femitpc is enabled, emit
      2 PC values: first the address of the instruction, then the address being
      jumped to. \\
      \hline
      {\tt jalr} & If \Femitbranch is disabled but \Femitpc is enabled, emit 2 PC
      values: first the address of the instruction, then the address being
      jumped to. Otherwise, if \Femitstoredata is enabled emit just the
      destination PC. \\
      \hline
      BRANCH & If \Femitbranch is enabled, emit either Branch Taken or Branch
      Not Taken.  Otherwise if \Femitpc is enabled and the branch is taken,
      emit 2 PC values: first the address of the branch, then the address being
      branched to. \\
      \hline
      LOAD & If \Femitloadaddr is enabled, emit the address.  If
      \Femitloaddata is enabled, emit the data that was loaded. \\
      \hline
      STORE & If \Femitstoreaddr is enabled, emit the address. If
      \Femitstoredata is enabled, emit the data that is stored. \\
      \hline
      Traps & {\tt scall}, {\tt sbreak}, {\tt ecall}, {\tt ebreak}, and {\tt
      eret} emit the same as if they were {\tt jal} instructions. In addition they
      also emit a Privilege Level sequence. \\
      \hline
      Interrupts & Emit PC (if enabled) of the last instruction executed.  Emit
      Privilege Level (if enabled).  Finally emit the new PC (if enabled). \\
      \hline
      CSR instructions & For reads emit Load Data (if enabled). For writes emit
      Store Data (if enabled). \\
      \hline
      Data Dropped & After packet sequences are dropped because data is
      generated too quickly, Trace Disabled must be emitted. It's not necessary
      to follow that up with a Trace Enabled sequence. \\
      \hline
   \end{tabulary}
\end{table}

\subsection{Synchronization}

If a trace buffer wraps, it is no longer clear what in the buffer is a header
and what isn't. To guarantee that a trace decoder can sync up easily, each
trace buffer must have 8 synchronization points, spaced evenly throughout the
buffer, with the first one at the very start of the buffer. A synchronization
point is simply an address where there is guaranteed to be a sequence header.
To make this happen, the trace source can insert a number of Nop headers into
the sequence just before writing to the synchronization point.

Aside from synchronizing a place in the data stream, it's also necessary to
send a full PC, Read Address, Write Address, and Timestamp in order for those
to be fully decoded. Ideally that happens the first time after every
synchronization point, but bandwidth might prevent that. A trace source should
attempt to send one full value for each of these (assuming they're enabled)
soon after each synchronization point.

\subsection{Trace Registers}

\input{trace_registers.tex}
