\section{About This Document}
\label{sec:about}

\subsection{Structure}

This document contains two parts. The main part of the document is the
specification, which is given in the numbered sections. The second part
of the document is a set of  appendices. The information
in the appendix is intended to clarify and provide examples, but is
not part of the actual specification.

\subsection{Terminology}

A \emph{platform} is a single integrated circuit consisting of one or more
\emph{components}.
Some components may be RISC-V cores, while others may have a different
function. Typically they will all be connected to a single system bus.
A single RISC-V core contains one or more hardware threads, called
\emph{harts}.

\subsection{Register Definitions}

All register definitions in this document follow the format shown in Section~\ref{shortname}.
A simple graphic shows which fields are in the register. The
upper and lower bit indices are shown to the top left and top right of each
field. The total number of bits in the field are shown below it.

After the graphic follows a table which for each field lists its name,
description, allowed accesses, and reset value. The allowed accesses are listed
in Table~\ref{tab:access}.

\begin{table}[htp]
    \centering
    \caption{Register Access Abbreviations}
    \label{tab:access}
    \begin{tabulary}{\textwidth}{|l|L|}
        \hline
        R & Read-only. \\
        \hline
        R/W & Read/Write. \\
        \hline
        R/W0 & Read/Write. Only writing 0 has an effect.  \\
        \hline
        R/W1 & Read/Write. Only writing 1 has an effect.  \\
        \hline
        W & Write-only. When read this field returns 0. \\
        \hline
        W1 & Write-only. Only writing 1 has an effect. \\
        \hline
    \end{tabulary}
\end{table}

\input{sample_registers.tex}

