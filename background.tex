\section{Background}
\label{sec:background}

There are two forms of external debugging. The first is \emph{halt mode}
debugging, where an external debugger will halt some or all components of a
platform and inspect them while they are in stasis. Then the debugger can allow
the hardware to either perform a single step or to run freely.

The second is \emph{run mode} debugging.  In this mode there is a software
debug agent running on a component (eg.  triggered by a timer interrupt on a
RISC-V core) which communicates with a debugger without halting the component.
This is essential if the component is controlling some real-time system (like a
hard drive) where halting the component might lead to physical damage. It
requires more software support (both on the chip as well as on the debug
client).  For this use case the debug interface may include simple serial
ports.

A third use for the external debug interface is to use it as a general
transport for a component to communicate with the outside world. For instance,
it could be used to implement a serial interface that firmware could use to
provide a simple CLI. This can use the same serial ports used for run-mode
debugging.
