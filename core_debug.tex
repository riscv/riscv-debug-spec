\chapter{RISC-V Debug}
\label{sec:core_debug}

Modifications to the RISC-V core to support debug are kept to a minimum.  There
is a special execution mode (Debug Mode) and a few extra CSRs. The DM takes care
of the rest.

\section{Debug Mode} \label{debugmode}

Debug Mode is a special processor mode used only when the core is halted for
external debugging. How Debug Mode is entered is implementation-specific.

\begin{steps}{When executing code from the Program Buffer, the processor stays
    in Debug Mode and the following apply:}
\item All operations happen in machine mode.
\item \Fmprv in \Rmstatus is ignored.
\item All interrupts are masked.
\item Exceptions don't update any registers.  That includes {\tt cause}, {\tt
    epc}, {\tt badaddr}, {\tt dpc}, and \Rmstatus. They do end execution of the
    Program Buffer.
\item No action is taken if a trigger matches.
\item Trace is disabled.
\item Counters may be stopped, depending on \Fstopcount in \Rdcsr.
\item Timers may be stopped, depending on \Fstoptime in \Rdcsr.
\item The {\tt wfi} instruction acts as a {\tt nop}.
\item Almost all instructions that change the privilege level have undefined
    behavior.  This includes {\tt ecall}, {\tt mret}, {\tt hret}, {\tt sret},
    and {\tt uret}.  (To change the privilege level, the debugger can write
    \Fprv in \Rdcsr.) The exception is {\tt ebreak}. When that is executed in
    Debug Mode, it halts the processor again but without updating \Rdpc.
\end{steps}

\section{Load-Reserved/Store-Conditional Instructions}

The reservation registered by an {\tt lr} instruction on a memory address may
be lost when entering Debug Mode or while in Debug Mode.  This means that there
may be no forward progress if Debug Mode is entered between {\tt lr} and {\tt
sc} pairs.

\section{Reset}

If the halt signal is asserted when a core comes out of reset, the core must
enter Debug Mode before executing any instructions, but after performing any
initialization that would usually happen before the first instruction is
executed.

\section{Core Debug Registers} \label{debreg}

The Core Debug Registers must be implemented for each hart being debugged.

\input{core_registers.tex}
