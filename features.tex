\section{Supported Features}
\label{sec:features}

The debug interface described out here supports the following features:
\begin{enumerate}
   \item RV32, RV64, and future RV128 are all supported.
   \item Any hart in the platform can be independently debugged.
   \item Arbitrary instructions can be executed on a halted hart. That means no
       new debug functionality is needed when a core has custom instructions or
       registers, as long as there exist programs that can store those
       registers to memory.
   \item With optional extensions it becomes possible to halt a core very
       briefly and quickly execute a a few instructions before letting it
       resume.
   \item Memory is accessed through the core, so the debugger will see the same
       thing that the code that is executing sees.
   \item Optionally, a bus master can be implemented to allow memory access
       without involving any hart.
   \item Debugging can be supported over multiple transports.
   \item Code can be downloaded efficiently.
   \item Each hart can be debugged from the very first instruction executed.
   \item A RISC-V core can be halted when a software breakpoint instruction is
       executed.
   \item Hardware can step over any instruction.
   \item A RISC-V core can be halted when a trigger matches the PC, read/write
       address/data, or an instruction opcode.
   \item The debug module may implement serial ports which can be used for
       communication between debugger and monitor, or as a general protocol
       between debugger and application.
   \item The debugger doesn't need to know anything about the microarchitecture
       of the cores it is debugging.
   \item The spec can be implemented without touching a core's data path.
\end{enumerate}
