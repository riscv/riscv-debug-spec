\section{RISC-V Debug}
\label{sec:hart}

Modifications to the RISC-V core to support debug are kept to a minimum.  There
is a special execution mode (Halt Mode) and a few extra CSRs. The DM takes care
of the rest.

\subsection{Hart IDs}

External debug imposes a few limits on hart IDs. Every hart in the system must
have a unique ID. (There could be additional harts that reuse IDs, but only one
of the harts that share an ID can be debugged.) One of the harts must use ID 0.
The debugger needs this to access the config string to enumerate the remaining
harts in the system. Hart IDs should be less than 128 if the Debug Bus address
is 5 bits wide, or less than 1024 if that address is 6 or more bits wide.

\subsection{Halt Mode}

Halt Mode is a special processor mode used only when the core is halted for
external debugging.

\begin{steps}{To enter Halt Mode the hart:}
\item Saves \Rpc to \Rdpc.
\item Sets \Fcause in \Rdcsr.
\item Starts fetching instructions from the DM.
\end{steps}

\begin{steps}{While in Halt Mode:}
\item All instructions come from the DM.
\item All operations happen in machine mode.
\item \Fmprv in \Rmstatus is ignored.
\item All interrupts are masked. Whether slow watchdog timers (10s or longer)
    are masked is left to the implementation.
\item Exceptions don't update any registers.  That includes {\tt cause}, {\tt
    epc}, {\tt badaddr}, and \Rmstatus.  Instead exceptions just set \Fhmexc in
    \Rdcsr.
\item No trigger actions are taken.
\item Trace is disabled.
\item Cycle counters may be stopped, depending on \Fstopcycle in \Rdcsr.
\item Timers may be stopped, depending on \Fstoptime in \Rdcsr.
\item The {\tt wfi} instruction acts as a {\tt nop}.
\item Instructions that change the privilege level have undefined behavior.
    This includes {\tt ecall}, {\tt ebreak}, {\tt mret}, {\tt hret}, {\tt
    sret}, and {\tt uret}.  (To change the privilege level, the debugger can
    write \Fprv in \Rdcsr.)
\end{steps}

\subsection{Load-Reserved/Store-Conditional Instructions}

The reservation registered by an {\tt lr} instruction on a memory address may
be lost when entering Halt Mode or while in Halt Mode.  This means that there
may be no forward progress if Halt Mode is entered between {\tt lr} and {\tt
sc} pairs.

\subsection{Reset}

If the halt signal is asserted when a core comes out of reset, the core must
enter Debug Mode before executing any instructions, but after performing any
initialization that would usually happen before the first instruction is
executed.

\subsection{Core Debug Registers} \label{debreg}

The Core Debug Registers must be implemented for each hart being debugged.

\input{core_registers.tex}
