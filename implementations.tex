\chapter{Hardware Implementations}
\label{sec:implementations}

Below are two possible implementations. A designer could choose one, mix and
match, or come up with their own design.

\section{Direct}

Halting happens by inhibiting instruction fetching in Debug Mode.

Muxes on the register file(s) allow for implementing those abstract commands.
(TODO: Should we spec a suggested way of communicating between DM and hart, so
that a reusable DM can be implemented?)

To execute a program buffer, force \Rpc to the start of it, and re-enable
instruction fetching (while staying in Debug Mode) until {\tt ebreak} is
encountered.

\Rdpc is not physically a CSR. Instead, accessing it directly access \Rpc.

\section{Plain Exception}

In this implementation, Debug Mode acts more like a real exception, jumping to a
memory region that is serviced by the DM. When taking this exception, \Rpc is
saved to \Rdpc. To allow the DM to individually control one out of several
halted harts, each hart jumps to a hard-coded unique address.

\Rdatazero etc. are mapped into regular memory at address 0x400. The important
property of that address is that it's reachable relative to \Rzero. The exact
address is an implementation detail that a debugger must not rely on.

When first halting, the code there looks like this:
\begin{minted}{gas}
dm_park_hart_0:
        nop
        j       dm_park_hart_0

dm_exception_hart_0:
        j       dm_park_hart_0
\end{minted}

The DM assumes that every instruction fetched from it is executed.

To implement the abstract 32-bit GPR access instructions, the {\tt nop} is
changed (for a single fetch) to {\tt lw <gpr>, 0x400(zero)} or {\tt sw
0x400(zero), <gpr>}. 64- and 128-bit accesses use {\tt ld}/{\tt sd} and {\tt
lq}/{\tt sq} respectively.

To execute the Program Buffer, the {\tt nop} is changed to {\tt j
dm\_program\_buffer}. When {\tt ebreak} is executed (indicating the end of the
Program Buffer code) the hart jumps back to its park loop. If an exception is
encountered, the hart jumps to its debug exception address, which in turn
contains a jump back to its park loop. The DM infers from the fetch at the park
address that there was an exception, and sets \Fcmderr appropriately.

To resume execution, the {\tt nop} is changed to {\tt dret}.  When {\tt dret}
is executed, \Rpc is restored from \Rdpc and normal execution resumes at the
privilege set by \Fprv.
