\chapter{Introduction}
\label{sec:intro}

When a design progresses from simulation to a hardware implementation, a users's
control and understanding of the system's current state drops dramatically.
To help bring up and debug low level software and hardware,
it is critical to have good debugging support built into the hardware.
When a robust OS is running on a core, software can handle many
debugging tasks. Howevever, in many scenarios, hardware support is essential.

This document outlines a standard architecture for external debug support 
on RISC-V platforms. This architecture allows a variety of implementations and
tradeoffs, which is complementary to the wide range of RISC-V implementations.
At the same time, this specification defines common interfaces to
allow debugging tools and components to target a variety of platforms based on the RISC-V ISA.

System designers may choose to add additional hardware debug support,
but this specification defines a standard interface for common
functionality.

\section{Terminology}

A \emph{platform} is a single integrated circuit consisting of one or more
\emph{components}. Some components may be RISC-V cores, while others may have a different
function. Typically they will all be connected to a single system bus.
A single RISC-V core contains one or more hardware threads, called
\emph{harts}.

\section{About This Document}

\subsection{Structure}

This document contains two parts. The main part of the document is the
specification, which is given in the numbered sections. The second part
of the document is a set of  appendices. The information
in the appendix is intended to clarify and provide examples, but is
not part of the actual specification.


\subsection{Register Definition Format}

All register definitions in this document follow the format shown in Section~\ref{shortname}.
A simple graphic shows which fields are in the register. The
upper and lower bit indices are shown to the top left and top right of each
field. The total number of bits in the field are shown below it.

After the graphic follows a table which for each field lists its name,
description, allowed accesses, and reset value. The allowed accesses are listed
in Table~\ref{tab:access}.

\begin{table}[htp]
    \centering
    \caption{Register Access Abbreviations}
    \label{tab:access}
    \begin{tabulary}{\textwidth}{|l|L|}
        \hline
        R & Read-only. \\
        \hline
        R/W & Read/Write. \\
        \hline
        R/W0 & Read/Write. Only writing 0 has an effect.  \\
        \hline
        R/W1 & Read/Write. Only writing 1 has an effect.  \\
        \hline
        W & Write-only. When read this field returns 0. \\
        \hline
        W1 & Write-only. Only writing 1 has an effect. \\
        \hline
    \end{tabulary}
\end{table}

\input{sample_registers.tex}

%\section{Reading Order}

This section describes the minimal parts of the spec required to understand a
given piece of functionality. It's still best to read the whole thing, but it
might help you get started better than reading it all start to finish. No
matter what you want to learn about, it's best to read Section~\ref{overview}
first.

\subsubsection{Halt/Resume}

Sections \ref{dmi}, \ref{selectingharts}, \ref{haltcontrol}, \ref{dmcontrol}, \ref{dmstatus},
\ref{haltsum}, \ref{deb:halt}.

\subsubsection{Abstract Register Access}

Sections \ref{abstractcommands}, \ref{abstractcs}, \ref{command},
\ref{data0}, \ref{deb:abstractreg}.

\subsubsection{Program Buffer}

Sections \ref{programbuffer}, \ref{progbufcs}, \ref{progbuf0}, \ref{access
register}, \ref{debugmode}, \ref{deb:regprogbuf}, \ref{deb:mrprogbuf}.

\subsubsection{JTAG Debug Transport Module}

Sections \ref{dtm}, \ref{jtagdtm}, \ref{dbusaccess}.

\subsubsection{System Bus Access}

Sections \ref{systembusaccess}, \ref{sbcs}, \ref{sbaddress0}, \ref{sbaddress1},
\ref{sbaddress2}, \ref{sbdata0}, \ref{sbdata1}, \ref{sbdata2}, \ref{sbdata3},
\ref{deb:mrsysbus}.

\subsubsection{Reset}

TODO

\subsubsection{Security}

TODO

\subsubsection{Triggers}

TODO

\subsubsection{Serial Ports}

TODO


