\section{JTAG Debug Transport Module} \label{sec:jtagdtm}

This Debug Transport Module is based around a normal JTAG Test Access Port
(TAP).  The JTAG TAP allows access to arbitrary JTAG registers by first
selecting one using the JTAG instruction register (IR), and then accessing it
through the JTAG data register (DR).

\subsection{JTAG Background}

JTAG refers to IEEE Std 1149.1-2013. It is a standard that defines test logic
that can be included in an integrated circuit to test the interconnections
between integrated circuits, test the integrated circuit itself, and observe or
modify circuit activity during the component's normal operation.
This specification uses the latter functionality.
The JTAG standard defines a Test Access Port (TAP) that
can be used to read and write a few custom registers, which can be used to
communicate with debug hardware in a component.

\subsection{JTAG DTM Registers}

JTAG TAPs used as a DTM must have an IR of at least 5 bits.
When the TAP is reset, IR must default to
00001, selecting the IDCODE instruction. A full list of JTAG registers along
with their encoding is in Table~\ref{dtmTable:jtagregisters}.
If the IR actually has more than 5 bits, then the encodings in
Table~\ref{dtmTable:jtagregisters} should be extended with 0's in their most
significant bits, except for the 0x1f encoding of BYPASS, which must be
extended with 1's in the most significant bits.
The only regular JTAG registers a debugger might use are BYPASS and IDCODE, but this
specification leaves IR space for many other standard JTAG instructions.
Unimplemented instructions must select the BYPASS register.

\input{jtag_registers.tex}

\subsection{JTAG Connector}

\subsubsection{JTAG Connector}

To make it easy to acquire debug hardware, this spec recommends a connector
that is compatible with the MIPI-10 .05 inch connector specification, as described
in MIPI Debug \& Trace Connector Recommendations, Version 1.20, 2 July 2021.

The connector has .05 inch spacing, gold-plated male header with .016 inch thick
hardened copper or beryllium bronze square posts (SAMTEC FTSH or equivalent).
Female connectors are compatible $20\mu m$ gold connectors.

Viewing the male header from above (the pins pointing at your eye), a target's
connector looks as it does in Table~\ref{tab:mipiten}.  The function of each pin
is described in Table~\ref{tab:pinout}.

\begin{table}[H]
    \centering
    \caption{MIPI 10-pin JTAG + nRESET Connector Diagram}
    \label{tab:mipiten}
    \begin{tabular}{|r|c|c|l|}
        \hline
        VREF DEBUG & 1 & 2 & TMS \\
        \hline
        GND & 3 & 4 & TCK \\
        \hline
        GND & 5 & 6 & TDO \\
        \hline
        GND or KEY & 7 & 8 & TDI \\
        \hline
        GND & 9 & 10 & nRESET \\
        \hline
    \end{tabular}
\end{table}

\begin{table}[htp]
    \centering
    \caption{JTAG Connector Pin Functions}
    \label{tab:pinout}
    \begin{tabulary}{\textwidth}{|c|L|}
      \hline
      VREF DEBUG & Reference voltage for logic high. \\
      \hline
      GND & Connected to ground. \\
      \hline
      TMS & JTAG TMS signal, driven by the debug adapter. \\
      \hline
      TCK & JTAG TCK signal, driven by the debug adapter. \\
      \hline
      TDO & JTAG TDO signal, driven by the target. \\
      \hline
      GND or KEY &
        This pin may be cut on the male and plugged on the female header to
        ensure the header is always plugged in correctly. It is, however,
        recommended to use this pin as an additional ground, to allow for
        fastest TCK speeds. A shrouded connector should be used to prevent the
        cable from being plugged in incorrectly. \\
      \hline
      TDI & JTAG TDI signal, driven by the debug adapter. \\
      \hline
      nRESET & Active-low reset signal, driven by the debug adapter.
        Asserting reset should reset any RISC-V cores as well as any other
        peripherals on the PCB. It should not reset the debug logic.  This pin
        is optional but strongly encouraged.

        If necessary, this pin could be used as nTRST instead.

        nRESET should never be connected to the TAP reset, otherwise the
        debugger might not be able to debug through a reset to discover the
        cause of a crash or to maintain execution control after the reset. \\
      \hline
      RTCK & Return test clock, driven by the target. A target may relay
        the TCK signal here once it has processed it, allowing a debugger to
        adjust its TCK frequency in response. \\
      \hline
      nTRST\_PD & Test reset pull-down (optional), driven by the debug
        adapter. Same function as nTRST, but with pull-down resistor on target.
        \\
      \hline
      nTRST & Test reset (optional), driven by the debug adapter.  Used to
        reset the JTAG TAP Controller. \\
      \hline
      TRIGIN & Not used by this specification, to be driven by debug
        adapter.  (Can be used for extended functions like UART or boot mode
        selection by some debug adapters). \\
      \hline
      TRIGOUT & Not used by this specification, driven by the target. \\
      \hline
    \end{tabulary}
\end{table}

If a hardware platform requires nTRST then it is permissible to reuse the nRESET pin as
the nTRST signal, resulting in a MIPI 10-pin JTAG + nTRST connector.

\subsubsection{Alternate JTAG Connector}

The MIPI-10 connector should provide plenty of signals for all modern hardware.
If a design does need more esoteric signals, then the
MIPI-20 connector should be used. Its physical connector is virtually identical
to MIPI-10, except that it's twice as long, supporting twice as many pins. Its
pinout is shown in Table~\ref{tab:mipitwenty}. The function of each pin
is described in Table~\ref{tab:pinout}.

\begin{table}[H]
    \centering
    \caption{MIPI 20-pin JTAG Connector Diagram}
    \label{tab:mipitwenty}
    \begin{tabular}{|r|c|c|l|}
        \hline
        VREF DEBUG & 1 & 2 & TMS \\
        \hline
        GND & 3 & 4 & TCK \\
        \hline
        GND & 5 & 6 & TDO \\
        \hline
        GND or KEY & 7 & 8 & TDI \\
        \hline
        GND & 9 & 10 & nRESET \\
        \hline
        GND & 11 & 12 & RTCK \\
        \hline
        GND & 13 & 14 & nTRST\_PD \\
        \hline
        GND & 15 & 16 & nTRST \\
        \hline
        GND & 17 & 18 & TRIGIN \\
        \hline
        GND & 19 & 20 & TRIGOUT \\
        \hline
    \end{tabular}
\end{table}

\subsection{cJTAG}

This spec does not have specific recommendations on how to use the cJTAG
protocol.

When implementing cJTAG access to a JTAG DTM, the MIPI 10-pin Narrow JTAG
connector should be used. Viewing the male header from above (the pins pointing
at your eye), a target's connector looks as it does in
Table~\ref{tab:mipicjtag}.

\begin{table}[htp]
    \centering
    \caption{MIPI 10-pin Narrow JTAG Connector Diagram}
    \label{tab:mipicjtag}
    \begin{tabular}{|r|c|c|l|}
        \hline
        VREF DEBUG & 1 & 2 & TMSC \\
        \hline
        GND & 3 & 4 & TCKC \\
        \hline
        GND & 5 & 6 & EXT \\
        \hline
        GND or KEY & 7 & 8 & nTRST\_PD \\
        \hline
        GND & 9 & 10 & nRESET \\
        \hline
    \end{tabular}
\end{table}
