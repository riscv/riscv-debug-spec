\documentclass[twoside,11pt]{article}
\setcounter{tocdepth}{4}
\setcounter{secnumdepth}{4}

\usepackage{enumitem}

\usepackage{xspace}
\newcommand{\defregname}[2]{\providecommand{#1}{{\tt #2}\xspace}}
\newcommand{\deffieldname}[2]{\providecommand{#1}{{\textsf #2}\xspace}}
\defregname{\Rsbdatazero}{sbdata0}
\deffieldname{\Fsbbusy}{sbbusy}
\deffieldname{\Fsbreadondata}{sbreadondata}
\deffieldname{\Fsbautoincrement}{sbautoincrement}

\newenvironment{steps}[1]
  {
     \vspace{1ex}
     \noindent #1
     \begin{enumerate}[nolistsep]
  }
  {
     \end{enumerate}
     \vspace{1ex}
 }

\input{vc.tex}

\begin{document}

\title{RISC-V External Debug Support Errata\\
\GITHash
}
\author{Editors: \\
Tim Newsome \textless tim@sifive.com\textgreater, SiFive, Inc. \\
Megan Wachs \textless megan@sifive.com\textgreater, SiFive, Inc.}
\date{\GITAuthorDate}
\maketitle

\section{Errata to RISC-V Debug Specification 0.13}

\subsection{\Rsbdatazero Reads Order of Operations}

The order of operations listed in Section 3.12.23, describing reads from
\Rsbdatazero, is incorrect. It should read:

\begin{steps}{Reads from this register start the following:}
    \item ``Return'' the data.
    \item Set \Fsbbusy.
    \item If \Fsbreadondata is set, perform another system bus read.
    \item If \Fsbautoincrement is set, increment {\tt sbaddress}.
    \item Clear \Fsbbusy.
\end{steps}

\end{document}
