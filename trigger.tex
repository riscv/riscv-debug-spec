\chapter{Trigger Module}
\label{sec:trigger}

Triggers can cause a breakpoint exception, entry into Debug Mode, or a trace action
without having to execute a special instruction. This makes them invaluable
when debugging code from ROM. They can trigger on execution of instructions at
a given memory address, or on the address/data in loads/stores.  These are all
features that can be useful without having the Debug Module present, so the
Trigger Module is broken out as a separate piece that can be implemented
separately.

A hart can be compliant with this specification without implementing any
trigger functionality at all, but if it is implemented then it must conform to
this section. If triggers aren't implemented, the CSRs may not exist at all and
accessing them results in an illegal instruction exception.

Triggers do not fire while in Debug Mode.

\begin{steps}{Each trigger may support a variety of features. A debugger can
    build a list of all triggers and their features as follows:}
\item Write 0 to \Rtselect. If this results in an illegal instruction
    exception, then there are no triggers implemented.
\item Read back \Rtselect and check that it contains the written value. If not,
    exit the loop.
\item Read \Rtinfo.
\item If that caused an exception, the debugger must read \Rtdataone to
    discover the type. (If \Ftype is 0, this trigger doesn't exist. Exit the
    loop.)
\item If \Finfo is 1, this trigger doesn't exist. Exit the loop.
\item Otherwise, the selected trigger supports the types discovered in \Finfo.
\item Repeat, incrementing the value in \Rtselect.
\end{steps}

\begin{commentary}
    The above algorithm reads back \Rtselect so that implementations which have
    $2^n$ triggers only need to implement $n$ bits of \Rtselect.

    The algorithm checks \Rtinfo and \Ftype in case the implementation has $m$
    bits of \Rtselect but fewer than $2^m$ triggers.
\end{commentary}

It is possible for a trigger with the ``enter Debug Mode'' action (1) and another
trigger with the ``raise a breakpoint exception'' action (0) to fire at the same
time. The preferred behavior is to have both actions take place.  It is
implementation-dependent which of the two happens first.  This ensures both
that the presence of an external debugger doesn't affect execution and that a
trigger set by user code doesn't affect the external debugger. If this is not
implemented, then the hart must enter Debug Mode and ignore the breakpoint
exception. In the latter case, \Fhit of the trigger whose action is 0 must still
be set, giving a debugger an opportunity to handle this case. What happens with
trace actions when triggers with different actions are also firing is left to
the trace specification.

\section{Native M-Mode Triggers}
\label{sec:mmtrigger}

Triggers can be used for native debugging. On a fully featured system triggers
will be set using \Fu or \Fs, and when firing they can cause a breakpoint exception
to trap to a more privileged mode. It is possible to set triggers natively to
fire in M mode as well. In that case there is no higher privilege mode to trap
to. When such a trigger causes a breakpoint exception while already in a trap
handler, this will leave the system unable to resume normal execution.

On full-featured systems this is a remote corner case that can probably be
ignored. On systems that only implement M mode, however, it is recommended to
implement one of two solutions to this problem. This way triggers can be useful
for native debugging of even M mode code.

The simple solution is to have the hardware prevent triggers with action=0 from
firing while in M mode and while \Fmie in \Rmstatus is 0. Its limitation is
that interrupts might be disabled at other times when a user might want
triggers to fire.

A more complex solution is to implement \Fmte and \Fmpte in \Rtcontrol. This
solution has the benefit that it only disables triggers during the trap
handler.

A user setting M mode triggers that cause breakpoint exceptions will have to be
aware of any problems that might come up with the particular system they are
working on.

\section{Trigger Registers}

These registers are CSRs, accessible using the RISC-V {\tt csr} opcodes and
optionally also using abstract debug commands.

Almost all trigger functionality is optional. All {\tt tdata} registers follow
write-any-read-legal semantics. If a debugger writes an unsupported
configuration, the register will read back a value that is supported (which may
simply be a disabled trigger).  This means that a debugger must always read
back values it writes to {\tt tdata} registers, unless it already knows already
what is
supported.  Writes to one {\tt tdata} register may not modify the contents of
other {\tt tdata} registers, nor the configuration of any trigger besides the
one that is currently selected.

\input{hwbp_registers.tex}
