\section{Trigger Module}
\label{sec:trigger}

Triggers can cause a debug exception, entry into Halt Mode, or a trace action
without having to execute a special instruction. This makes them invaluable
when debugging code from ROM. They can trigger on execution of instructions at
a given memory address, or on the address/data in loads/stores.  These are all
features that can be useful without having the Debug Module present, so the
Trigger Module is broken out as a separate piece that can be implemented
separately.

\begin{steps}{Each trigger may support a variety of features. A debugger can
    build a list of all triggers and their features as follows:}
\item Write 0 to \Rtselect.
\item Read back \Rtselect to confirm this trigger exists. If not, exit.
\item Read \Rtdataone, and possible \Rtdatatwo and \Rtdatathree depending on the
    trigger type.
\item If \Ftype in \Rtdataone was 0, then there are no more triggers.
\item Repeat, incrementing the value in \Rtselect.
\end{steps}

\subsection{Trigger Registers}

\input{hwbp_registers.tex}
